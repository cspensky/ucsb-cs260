%%%%%%%%%%%%%%%%%%%%%%%%%%%%%%%%%%%%%%%%%
% Structured General Purpose Assignment
% LaTeX Template
%
% This template has been downloaded from:
% http://www.latextemplates.com
%
% Original author:
% Ted Pavlic (http://www.tedpavlic.com)
%
% Note:
% The \lipsum[#] commands throughout this template generate dummy text
% to fill the template out. These commands should all be removed when 
% writing assignment content.
%
%%%%%%%%%%%%%%%%%%%%%%%%%%%%%%%%%%%%%%%%%

%----------------------------------------------------------------------------------------
%	PACKAGES AND OTHER DOCUMENT CONFIGURATIONS
%----------------------------------------------------------------------------------------

\documentclass{article}

\usepackage{fancyhdr} % Required for custom headers
\usepackage{lastpage} % Required to determine the last page for the footer
\usepackage{extramarks} % Required for headers and footers
\usepackage{graphicx} % Required to insert images
\usepackage{lipsum} % Used for inserting dummy 'Lorem ipsum' text into the template
\usepackage{amsfonts}
\usepackage{subfig}
\usepackage{graphicx}
\usepackage{caption}
\usepackage{amsmath}

%
%% Margins
%\topmargin=-0.45in
%\evensidemargin=0in
%\oddsidemargin=0in
%\textwidth=6.5in
%\textheight=9.0in
%\headsep=0.25in 
%
%\linespread{1.1} % Line spacing
%
%% Set up the header and footer
%\pagestyle{fancy}
%\lhead{\hmwkAuthorName} % Top left header
%\chead{\hmwkClass\ (\hmwkClassInstructor): \hmwkTitle} % Top center header
%\rhead{\firstxmark} % Top right header
%\lfoot{\lastxmark} % Bottom left footer
%\cfoot{} % Bottom center footer
%\rfoot{Page\ \thepage\ of\ \pageref{LastPage}} % Bottom right footer
%\renewcommand\headrulewidth{0.4pt} % Size of the header rule
%\renewcommand\footrulewidth{0.4pt} % Size of the footer rule
%
%\setlength\parindent{0pt} % Removes all indentation from paragraphs
%
%%----------------------------------------------------------------------------------------
%%	DOCUMENT STRUCTURE COMMANDS
%%	Skip this unless you know what you're doing
%%----------------------------------------------------------------------------------------
%
%% Header and footer for when a page split occurs within a problem environment
%\newcommand{\enterProblemHeader}[1]{
%\nobreak\extramarks{#1}{#1 continued on next page\ldots}\nobreak
%\nobreak\extramarks{#1 (continued)}{#1 continued on next page\ldots}\nobreak
%}
%
%% Header and footer for when a page split occurs between problem environments
%\newcommand{\exitProblemHeader}[1]{
%\nobreak\extramarks{#1 (continued)}{#1 continued on next page\ldots}\nobreak
%\nobreak\extramarks{#1}{}\nobreak
%}
%
%\setcounter{secnumdepth}{0} % Removes default section numbers
%\newcounter{homeworkProblemCounter} % Creates a counter to keep track of the number of problems
%
%\newcommand{\homeworkProblemName}{}
%\newenvironment{homeworkProblem}[1][Problem \arabic{homeworkProblemCounter}]{ % Makes a new environment called homeworkProblem which takes 1 argument (custom name) but the default is "Problem #"
%\stepcounter{homeworkProblemCounter} % Increase counter for number of problems
%\renewcommand{\homeworkProblemName}{#1} % Assign \homeworkProblemName the name of the problem
%\section{\homeworkProblemName} % Make a section in the document with the custom problem count
%\enterProblemHeader{\homeworkProblemName} % Header and footer within the environment
%}{
%\exitProblemHeader{\homeworkProblemName} % Header and footer after the environment
%}
%
%\newcommand{\problemAnswer}[1]{ % Defines the problem answer command with the content as the only argument
%\noindent\framebox[\columnwidth][c]{\begin{minipage}{0.98\columnwidth}#1\end{minipage}} % Makes the box around the problem answer and puts the content inside
%}
%
%\newcommand{\homeworkSectionName}{}
%\newenvironment{homeworkSection}[1]{ % New environment for sections within homework problems, takes 1 argument - the name of the section
%\renewcommand{\homeworkSectionName}{#1} % Assign \homeworkSectionName to the name of the section from the environment argument
%\subsection{\homeworkSectionName} % Make a subsection with the custom name of the subsection
%\enterProblemHeader{\homeworkProblemName\ [\homeworkSectionName]} % Header and footer within the environment
%}{
%\enterProblemHeader{\homeworkProblemName} % Header and footer after the environment
%}
   
%----------------------------------------------------------------------------------------
%	NAME AND CLASS SECTION
%----------------------------------------------------------------------------------------

\newcommand{\hmwkTitle}{Assignment\ \#4\\ Regular Expression} % Assignment title
\newcommand{\hmwkDueDate}{Tuesday,\ November\ 24,\ 2015} % Due date
\newcommand{\hmwkClass}{CS\ 260} % Course/class
\newcommand{\hmwkClassInstructor}{Prof. Hardekopf} % Teacher/lecturer
\newcommand{\hmwkAuthorName}{Chad Spensky} % Your name

%----------------------------------------------------------------------------------------
%	TITLE PAGE
%----------------------------------------------------------------------------------------

\title{
\vspace{2in}
\textmd{\textbf{\hmwkClass:\ \hmwkTitle}}\\
\normalsize\vspace{0.1in}\small{Due\ on\ \hmwkDueDate}\\
\vspace{0.1in}\large{\textit{\hmwkClassInstructor}}
\vspace{3in}
}

\author{\textbf{\hmwkAuthorName}}
\date{} % Insert date here if you want it to appear below your name

%----------------------------------------------------------------------------------------
\newcommand{\powerset}[1]{\mathbb{P}(#1)}

\newcommand{\regExpSet}[1]{\mathbf{regExpSet}(#1)}
\newcommand{\genRegExp}[1]{\mathbf{genRegExp}(#1)}
\begin{document}


\maketitle
\newpage



\section{Abstraction Domain Lattice}
Let our original lattice $L = (\powerset{S}, \leq)$ and our abstract-domain lattice $L^\# = (\regExpSet{x}, x \in \powerset{S \cup \{|,*,+,(,)\} } \cup \{ \top,\bot \},\sqsubseteq)$ where $S$ is the set of all strings (including the empty string) where $\top = S$ and $\bot$ is \emph{undefined}. For connivence let all regular expressions, $expr$, denote the set of all possible strings that can be constructed from the expression, $\regExpSet{expr}$, and  $s$ denote the singleton set $\{s\}$, thus $\sqsubseteq = \subseteq$ in our lattice.  Intuitively, $\regExpSet{s} = s$.  Also, let $\genRegExp{\{a,b,c\}}$ be the function that will generate the most concise regular expression to that contains all the elements in the set (Note that $\top$ will never be returned because of the $|$ operator, i.e. a regular expression can always be generated).
\[
 \alpha(x): L \rightarrow L^\# =
  \begin{cases} 
      \hfill \bot    \hfill & \text{ if $x$ is \{\}} \\
      \hfill \genRegExp{s} \hfill & \text{Otherwise} \\
  \end{cases}
\]

\paragraph{Meet ($x \in L^\#$)}
\begin{itemize}
	\item $\bot \sqcap x = \bot, \forall x$
	\item $\top \sqcap x = x, \forall x$
	\item $x \sqcap y = x, \genRegExp{\regExpSet{x} \cap \regExpSet{y}}$
\end{itemize}
\paragraph{Join ($x \in L^\#$)}
\begin{itemize}
	\item $\bot \sqcup x = x, \forall x$
	\item $\top \sqcup x = \top, \forall x$
	\item $x \sqcup y = \genRegExp{\regExpSet{x} \cup \regExpSet{y}}$
\end{itemize}

The lattice is infinite, and also of infinite height because of the inclusion of the $|$ operator. The $+$ can only express strings which have common subsequences, and are not expressive enough to grow infinitely since our $\genRegExp{}$ will always return the most concise possible regular expression.  Since the $|$ operator is our pain point, our widening operator must remove the $|$'s.  Let our widening operator be defined as follows:
\[
 x \bigtriangledown_{lim} y : L^\# \times L^\# \rightarrow L^\# =
  \begin{cases} 
      \hfill x \sqcup y    \hfill & \text{ \# of $|$'s in $x \sqcup y \leq lim$} \\
      \hfill \top \hfill & \text{Otherwise} \\
  \end{cases}
\]
This definition trivially satisfies condition 1,  $x \sqsubseteq x \bigtriangledown_{lim} y, y \sqsubseteq x \bigtriangledown_{lim} y$.  Similarly, because of our definition, no chains can be strictly ascending, as they will be limited by $lim$.

\subsection{Monotone Operators}
\begin{figure*}
\centering
\subfloat[Concatenation (+)]{
    \begin{tabular}{c|ccccc}
     			& $\bot$  	&	$y$ 	&$\top$ \\ \hline
    $\bot$  	&    $\bot$	&	$y$		&$\top$          \\
    $x$  		&   	$x$		&	$x||y$	&$\top$          \\
    $\top$  	&    $\top$	&	$\top$	&$\top$          \\
    \end{tabular}
}~~~~
\subfloat[Comparator ($\leq$)]{
    \begin{tabular}{c|ccccc}
     			& $\bot$  	&	$y$ 	&$\top$ \\ \hline
    $\bot$  	&    $T$		&	$T$		&$T$          \\
    $x$  		&   	$F$		&	$x \sqsubseteq y?$	&$T,F$          \\
    $\top$  	&    $F$		&	$T,F$	&$T,F$          \\
    \end{tabular}
}
\caption{Arithmetic Tables}
\label{fig:table}
\end{figure*}


\paragraph{Concatenation (+)}
For + to be monotone the following must hold: $x+y \leq x'+y' \Rightarrow  \alpha(x)+\alpha(y) \sqsubseteq \alpha(x')+\alpha(y')$, where $x,y,x',y' \in \powerset{S}$ and $a \leq b \Rightarrow \mathbf{substring}(a,b)$ for $a,b \in S$ and $x+y \Rightarrow \{x_{i}y_i\} \forall x_i \in x, y_i \in y$.  Similarly $x \leq y \Rightarrow \mathbf{substring}(x_i, y_i) \forall x_i \in x, y_i \in y$.  In all cases $|x| = |y|$ and $|x'| = |y'|$.

\begin{itemize}
	\item For case where $x = y = \text{``''}$, this holds trivially.
	\item Otherwise, we need to show that $\genRegExp{x}||\genRegExp{y} \sqsubseteq  \genRegExp{x'}||\genRegExp{y'}$.  We know that $xy \leq x'y''$, i.e. that $xy$ is a substring of $x'y'$.  Thus since $\genRegExp{}$ returns the most concise regular expression that represents all $a \in x$, $x \subseteq \regExpSet{\genRegExp{x}}$. Thus, $xy \subseteq \regExpSet{\genRegExp{xy}}$ and $x'y' \subseteq \regExpSet{\genRegExp{x'y'}}$.  Since each regular expression necessarily contained the initial set, concatenating the expressions will still satisfy the condition of each, by taking the two representative sets and concatenating the strings of each. If there exists an element $z \in \regExpSet{\genRegExp{x \cup y}}, z \notin \regExpSet{\genRegExp{x' \cup y'}}$, this would imply that there are elements that are represented by $\genRegExp{xy}$ that are not represented by $\genRegExp{x' y'}$, which we know can't happen since $xy \leq x'y'$.  Therefore our assertion must hold.
\end{itemize}
$\square$

\paragraph{Comparison ($\leq$)}
The same logic follows for $\leq$. For $\leq$ to be monotone the following must hold: $x \leq y \Rightarrow  \alpha(x) \sqsubseteq \alpha(y)$, where $x,y \in \powerset{S}$ and $x \leq y \Rightarrow x \subseteq y$.
\begin{itemize}
	\item For case where $x = y = \{\}$, this holds trivially.
	\item Otherwise, we must show that $\genRegExp{x} \sqsubseteq \genRegExp{y}$. Since $\genRegExp{}$ returns the most concise regular expression, we not that $x \subseteq \regExpSet{\genRegExp{x}}$.  Thus $\regExpSet{\genRegExp{x}} \sqsubseteq \regExpSet{\genRegExp{y}}$, which by similar logic to above, i.e. $z \in \regExpSet{\genRegExp{x}} \Rightarrow \regExpSet{\genRegExp{y}}$, that is the regular expression for $\alpha(y)$ must at least express everything can be expressed by $\alpha(x)$.
\end{itemize}
$\square$

\subsection{Galois Connection}
\[
 \gamma(\hat{x}): L^\# \rightarrow L =
  \begin{cases} 
      \hfill \{\}    \hfill & \text{ if $\hat{x}$ is $\bot$} \\
      \hfill  \regExpSet{x} \hfill & \text{ if $\hat{x}$ is a regular expression} \\
      \hfill S \hfill & \text{ if $\hat{x}$ is $\top$}
  \end{cases}
\]
We must show that $\alpha(x) \sqsubseteq \hat{x} \iff x \subseteq \gamma(\hat{x})$.
\\\\
$\alpha(x) \sqsubseteq \hat{x} \Rightarrow x \subseteq \gamma(\hat{x})$:
\begin{itemize}
	\item If  $x = \{\}$, this holds trivially.
	\item Otherwise, either $\hat{x} = \top$, which holds trivially, or $\hat{x}$ is a regular expression. Since our regular expression are the most concise possible expressions that represent all elements the input set, we know that $x \subseteq \gamma(\hat{x})$ must be true. 
\end{itemize}
$ x \subseteq \gamma(\hat{x}) \Rightarrow \alpha(x) \sqsubseteq \hat{x}$:
\begin{itemize}
	\item If  $x = \{\}$, this holds trivially.
	\item Otherwise, every element in $x$ can be represented by $\hat{x}$, thus $\alpha(x) \sqsubseteq \hat{x}$ must be true since, $\alpha(x)$ returns the most precise regular expression representing all of $x$.  Any other conclusion would be a contradiction to our definition.
	\end{itemize}
$\square$

\subsection{Soundness}
Because a Galois connection exists, our approximation is sound.  However, because we are using a widening operator, we lose precision.  However, in the best case, i.e. we never hit our limit of $|$'s, we will actually return the exact input.  Otherwise, we will only lose precision on those few cases.








\end{document}