%%%%%%%%%%%%%%%%%%%%%%%%%%%%%%%%%%%%%%%%%
% Structured General Purpose Assignment
% LaTeX Template
%
% This template has been downloaded from:
% http://www.latextemplates.com
%
% Original author:
% Ted Pavlic (http://www.tedpavlic.com)
%
% Note:
% The \lipsum[#] commands throughout this template generate dummy text
% to fill the template out. These commands should all be removed when 
% writing assignment content.
%
%%%%%%%%%%%%%%%%%%%%%%%%%%%%%%%%%%%%%%%%%

%----------------------------------------------------------------------------------------
%	PACKAGES AND OTHER DOCUMENT CONFIGURATIONS
%----------------------------------------------------------------------------------------

\documentclass{article}

\usepackage{fancyhdr} % Required for custom headers
\usepackage{lastpage} % Required to determine the last page for the footer
\usepackage{extramarks} % Required for headers and footers
\usepackage{graphicx} % Required to insert images
\usepackage{lipsum} % Used for inserting dummy 'Lorem ipsum' text into the template
\usepackage{amsfonts}
\usepackage{subfig}
\usepackage{graphicx}
\usepackage{caption}
\usepackage{amsmath}

%
%% Margins
%\topmargin=-0.45in
%\evensidemargin=0in
%\oddsidemargin=0in
%\textwidth=6.5in
%\textheight=9.0in
%\headsep=0.25in 
%
%\linespread{1.1} % Line spacing
%
%% Set up the header and footer
%\pagestyle{fancy}
%\lhead{\hmwkAuthorName} % Top left header
%\chead{\hmwkClass\ (\hmwkClassInstructor): \hmwkTitle} % Top center header
%\rhead{\firstxmark} % Top right header
%\lfoot{\lastxmark} % Bottom left footer
%\cfoot{} % Bottom center footer
%\rfoot{Page\ \thepage\ of\ \pageref{LastPage}} % Bottom right footer
%\renewcommand\headrulewidth{0.4pt} % Size of the header rule
%\renewcommand\footrulewidth{0.4pt} % Size of the footer rule
%
%\setlength\parindent{0pt} % Removes all indentation from paragraphs
%
%%----------------------------------------------------------------------------------------
%%	DOCUMENT STRUCTURE COMMANDS
%%	Skip this unless you know what you're doing
%%----------------------------------------------------------------------------------------
%
%% Header and footer for when a page split occurs within a problem environment
%\newcommand{\enterProblemHeader}[1]{
%\nobreak\extramarks{#1}{#1 continued on next page\ldots}\nobreak
%\nobreak\extramarks{#1 (continued)}{#1 continued on next page\ldots}\nobreak
%}
%
%% Header and footer for when a page split occurs between problem environments
%\newcommand{\exitProblemHeader}[1]{
%\nobreak\extramarks{#1 (continued)}{#1 continued on next page\ldots}\nobreak
%\nobreak\extramarks{#1}{}\nobreak
%}
%
%\setcounter{secnumdepth}{0} % Removes default section numbers
%\newcounter{homeworkProblemCounter} % Creates a counter to keep track of the number of problems
%
%\newcommand{\homeworkProblemName}{}
%\newenvironment{homeworkProblem}[1][Problem \arabic{homeworkProblemCounter}]{ % Makes a new environment called homeworkProblem which takes 1 argument (custom name) but the default is "Problem #"
%\stepcounter{homeworkProblemCounter} % Increase counter for number of problems
%\renewcommand{\homeworkProblemName}{#1} % Assign \homeworkProblemName the name of the problem
%\section{\homeworkProblemName} % Make a section in the document with the custom problem count
%\enterProblemHeader{\homeworkProblemName} % Header and footer within the environment
%}{
%\exitProblemHeader{\homeworkProblemName} % Header and footer after the environment
%}
%
%\newcommand{\problemAnswer}[1]{ % Defines the problem answer command with the content as the only argument
%\noindent\framebox[\columnwidth][c]{\begin{minipage}{0.98\columnwidth}#1\end{minipage}} % Makes the box around the problem answer and puts the content inside
%}
%
%\newcommand{\homeworkSectionName}{}
%\newenvironment{homeworkSection}[1]{ % New environment for sections within homework problems, takes 1 argument - the name of the section
%\renewcommand{\homeworkSectionName}{#1} % Assign \homeworkSectionName to the name of the section from the environment argument
%\subsection{\homeworkSectionName} % Make a subsection with the custom name of the subsection
%\enterProblemHeader{\homeworkProblemName\ [\homeworkSectionName]} % Header and footer within the environment
%}{
%\enterProblemHeader{\homeworkProblemName} % Header and footer after the environment
%}
   
%----------------------------------------------------------------------------------------
%	NAME AND CLASS SECTION
%----------------------------------------------------------------------------------------

\newcommand{\hmwkTitle}{Assignment\ \#4\\ String Constant} % Assignment title
\newcommand{\hmwkDueDate}{Tuesday,\ November\ 24,\ 2015} % Due date
\newcommand{\hmwkClass}{CS\ 260} % Course/class
\newcommand{\hmwkClassInstructor}{Prof. Hardekopf} % Teacher/lecturer
\newcommand{\hmwkAuthorName}{Chad Spensky} % Your name

%----------------------------------------------------------------------------------------
%	TITLE PAGE
%----------------------------------------------------------------------------------------

\title{
\vspace{2in}
\textmd{\textbf{\hmwkClass:\ \hmwkTitle}}\\
\normalsize\vspace{0.1in}\small{Due\ on\ \hmwkDueDate}\\
\vspace{0.1in}\large{\textit{\hmwkClassInstructor}}
\vspace{3in}
}

\author{\textbf{\hmwkAuthorName}}
\date{} % Insert date here if you want it to appear below your name

%----------------------------------------------------------------------------------------
\newcommand{\powerset}[1]{\mathbb{P}(#1)}


\begin{document}


\maketitle
\newpage



\section{Abstraction Domain Lattice}
Let our original lattice $L = (\powerset{S}, \leq)$ and our abstract-domain lattice $L^\# = (S \cup \{ \top,\bot \},\sqsubseteq)$ where $S$ is the set of all strings (including the empty string) where $\top = S$ and $\bot$ is \emph{undefined}. Let $\bot \sqsubset s \sqsubset \top, \forall s \in S$, i.e. the lattice of height 3, infinite width. 
\[
 \alpha(x): L \rightarrow L^\# =
  \begin{cases} 
      \hfill \bot    \hfill & \text{ if $x$ is \{\}} \\
      \hfill s \hfill & \text{ if $x$ is \{s\}} \\
      \hfill \top \hfill & \text{otherwise}
  \end{cases}
\]

\paragraph{Meet}
\begin{itemize}
	\item $\bot \sqcap x = \bot, \forall x$
	\item $\top \sqcap x = x, \forall x$
	\item $x \sqcap y = \bot, \forall x,y \in S, x \neq y$
\end{itemize}
\paragraph{Join}
\begin{itemize}
	\item $\bot \sqcup x = x, \forall x$
	\item $\top \sqcup x = \top, \forall x$
	\item $x \sqcup y = \top, \forall x,y \in S, x \neq y$
\end{itemize}

The lattice is infinite, but of of finite height, therefore it \textbf{is noetherian}.

\subsection{Monotone Operators}
\begin{figure*}

\centering
\subfloat[Concatenation (+)]{
    \begin{tabular}{c|ccccc}
     			& $\bot$  	&	$y \in S$ 	& $\top$ \\ \hline
    $\bot$  		&    $\bot$	&	$y$		& $\top$          \\
    $x \in S$  	&   	$x$		&	$\top$	& $\top$          \\
    $\top$  		&    $\top$	&	$\top$	& $\top$          \\
    \end{tabular}
}~~~~~
\subfloat[Comparator ($\leq$)]{
 \begin{tabular}{c|ccccc}
   
     			& $\bot$  	&	$y \in S$ 	& $\top$ \\ \hline
    $\bot$  		&    $T$	&	$T$		& $T$          \\
    $x \in S$  	&   	$F$		&	$\mathsf{substring}(x,y)$	& $T,F$          \\
    $\top$  		&    $F$	&	$T,F$	& $T,F$          \\
    \end{tabular}
}
\caption{Arithmetic Tables}
\label{fig:table}
\end{figure*}


\paragraph{Concatenation (+)}
For + to be monotone the following must hold: $x+y \leq x'+y' \Rightarrow  \alpha(x)+\alpha(y) \sqsubseteq \alpha(x')+\alpha(y')$, where $x,y,x',y' \in \powerset{S}$ and $a \leq b \Rightarrow \mathbf{substring}(a,b)$ for $a,b \in S$ and $x+y \Rightarrow \{x_i+y_i\} \forall x_i \in x, y_i \in y$.  Similarly $x \leq y \Rightarrow \mathbf{substring}(x_i, y_i) \forall x_i \in x, y_i \in y$.  In all cases $|x| = |y|$ and $|x'| = |y'|$.

\begin{itemize}
	\item For case where $x = y = \{\}$, this holds trivially.
	\item Similarly for $|x| = |y| \geq 1$ and $|x'| = |y'| \geq 1$, since $\top \sqsubseteq \top$ will always be true.
\end{itemize}
$\square$

\paragraph{Comparison ($\leq$)}
The same logic follows for $\leq$. For $\leq$ to be monotone the following must hold: $x \leq y \Rightarrow  \alpha(x) \sqsubseteq \alpha(y)$, where $x,y \in \powerset{S}$ and $x \leq y \Rightarrow \mathbf{substring}(x_i,y_i) \forall x_i \in x, y_i \in y$.
\begin{itemize}
	\item For case where $x = y = \{\}$, this holds trivially.
	\item Similarly for $|x| = |y| = 1$, then the definitions are identical.
	\item Similarly for $|x| = |y| \geq 1$, since $\top \sqsubseteq \top$ will always be true.
\end{itemize}
$\square$

\subsection{Galois Connection}
\[
 \gamma(\hat{x}): L^\# \rightarrow L =
  \begin{cases} 
      \hfill \{\}    \hfill & \text{ if $\hat{x}$ is $\bot$} \\
      \hfill \{s\} \hfill & \text{ if $\hat{x}$ is s} \\
      \hfill S \hfill & \text{ if $\hat{x}$ is $\top$}
  \end{cases}
\]
We must show that $\alpha(x) \sqsubseteq \hat{x} \iff x \subseteq \gamma(\hat{x})$.
\\\\
$\alpha(x) \sqsubseteq \hat{x} \Rightarrow x \subseteq \gamma(\hat{x})$:
\begin{itemize}
	\item If  $x = \{\}$, this holds trivially.
	\item If $|x| = 1$, then either $\hat{x} \in S$, which holds trivially, or $\hat{x} = \top$, and $x \subseteq S, \forall x$ 
	\item For $|x| \geq 1$, $\hat{x} = \top$ must be true, and $x \subseteq S, \forall x$ 
\end{itemize}
$ x \subseteq \gamma(\hat{x}) \Rightarrow \alpha(x) \sqsubseteq \hat{x}$:
\begin{itemize}
	\item If  $x = \{\}$, this holds trivially.
	\item If $|x| = 1$, then either $\hat{x} \in S$, which holds trivially, or $\hat{x} = \top$, and $\alpha(x) \sqsubseteq \top, \forall x$ 
	\item For $|x| \geq 1$, $\hat{x} = \top$ must be true, and $\alpha(x) \subseteq \top, \forall x$ 
\end{itemize}
$\square$

\subsection{Soundness}
Because a Galois connection exists, our approximation is both sound and precise.








\end{document}