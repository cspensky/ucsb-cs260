%%%%%%%%%%%%%%%%%%%%%%%%%%%%%%%%%%%%%%%%%
% Structured General Purpose Assignment
% LaTeX Template
%
% This template has been downloaded from:
% http://www.latextemplates.com
%
% Original author:
% Ted Pavlic (http://www.tedpavlic.com)
%
% Note:
% The \lipsum[#] commands throughout this template generate dummy text
% to fill the template out. These commands should all be removed when 
% writing assignment content.
%
%%%%%%%%%%%%%%%%%%%%%%%%%%%%%%%%%%%%%%%%%

%----------------------------------------------------------------------------------------
%	PACKAGES AND OTHER DOCUMENT CONFIGURATIONS
%----------------------------------------------------------------------------------------

\documentclass{article}

\usepackage{fancyhdr} % Required for custom headers
\usepackage{lastpage} % Required to determine the last page for the footer
\usepackage{extramarks} % Required for headers and footers
\usepackage{graphicx} % Required to insert images
\usepackage{lipsum} % Used for inserting dummy 'Lorem ipsum' text into the template
\usepackage{amsfonts}
\usepackage{subfig}
\usepackage{graphicx}
\usepackage{caption}

%
%% Margins
%\topmargin=-0.45in
%\evensidemargin=0in
%\oddsidemargin=0in
%\textwidth=6.5in
%\textheight=9.0in
%\headsep=0.25in 
%
%\linespread{1.1} % Line spacing
%
%% Set up the header and footer
%\pagestyle{fancy}
%\lhead{\hmwkAuthorName} % Top left header
%\chead{\hmwkClass\ (\hmwkClassInstructor): \hmwkTitle} % Top center header
%\rhead{\firstxmark} % Top right header
%\lfoot{\lastxmark} % Bottom left footer
%\cfoot{} % Bottom center footer
%\rfoot{Page\ \thepage\ of\ \pageref{LastPage}} % Bottom right footer
%\renewcommand\headrulewidth{0.4pt} % Size of the header rule
%\renewcommand\footrulewidth{0.4pt} % Size of the footer rule
%
%\setlength\parindent{0pt} % Removes all indentation from paragraphs
%
%%----------------------------------------------------------------------------------------
%%	DOCUMENT STRUCTURE COMMANDS
%%	Skip this unless you know what you're doing
%%----------------------------------------------------------------------------------------
%
%% Header and footer for when a page split occurs within a problem environment
%\newcommand{\enterProblemHeader}[1]{
%\nobreak\extramarks{#1}{#1 continued on next page\ldots}\nobreak
%\nobreak\extramarks{#1 (continued)}{#1 continued on next page\ldots}\nobreak
%}
%
%% Header and footer for when a page split occurs between problem environments
%\newcommand{\exitProblemHeader}[1]{
%\nobreak\extramarks{#1 (continued)}{#1 continued on next page\ldots}\nobreak
%\nobreak\extramarks{#1}{}\nobreak
%}
%
%\setcounter{secnumdepth}{0} % Removes default section numbers
%\newcounter{homeworkProblemCounter} % Creates a counter to keep track of the number of problems
%
%\newcommand{\homeworkProblemName}{}
%\newenvironment{homeworkProblem}[1][Problem \arabic{homeworkProblemCounter}]{ % Makes a new environment called homeworkProblem which takes 1 argument (custom name) but the default is "Problem #"
%\stepcounter{homeworkProblemCounter} % Increase counter for number of problems
%\renewcommand{\homeworkProblemName}{#1} % Assign \homeworkProblemName the name of the problem
%\section{\homeworkProblemName} % Make a section in the document with the custom problem count
%\enterProblemHeader{\homeworkProblemName} % Header and footer within the environment
%}{
%\exitProblemHeader{\homeworkProblemName} % Header and footer after the environment
%}
%
%\newcommand{\problemAnswer}[1]{ % Defines the problem answer command with the content as the only argument
%\noindent\framebox[\columnwidth][c]{\begin{minipage}{0.98\columnwidth}#1\end{minipage}} % Makes the box around the problem answer and puts the content inside
%}
%
%\newcommand{\homeworkSectionName}{}
%\newenvironment{homeworkSection}[1]{ % New environment for sections within homework problems, takes 1 argument - the name of the section
%\renewcommand{\homeworkSectionName}{#1} % Assign \homeworkSectionName to the name of the section from the environment argument
%\subsection{\homeworkSectionName} % Make a subsection with the custom name of the subsection
%\enterProblemHeader{\homeworkProblemName\ [\homeworkSectionName]} % Header and footer within the environment
%}{
%\enterProblemHeader{\homeworkProblemName} % Header and footer after the environment
%}
   
%----------------------------------------------------------------------------------------
%	NAME AND CLASS SECTION
%----------------------------------------------------------------------------------------

\newcommand{\hmwkTitle}{Assignment\ \#3} % Assignment title
\newcommand{\hmwkDueDate}{Tuesday,\ November\ 3,\ 2015} % Due date
\newcommand{\hmwkClass}{CS\ 260} % Course/class
\newcommand{\hmwkClassInstructor}{Prof. Hardekopf} % Teacher/lecturer
\newcommand{\hmwkAuthorName}{Chad Spensky} % Your name

%----------------------------------------------------------------------------------------
%	TITLE PAGE
%----------------------------------------------------------------------------------------

\title{
\vspace{2in}
\textmd{\textbf{\hmwkClass:\ \hmwkTitle}}\\
\normalsize\vspace{0.1in}\small{Due\ on\ \hmwkDueDate}\\
\vspace{0.1in}\large{\textit{\hmwkClassInstructor}}
\vspace{3in}
}

\author{\textbf{\hmwkAuthorName}}
\date{} % Insert date here if you want it to appear below your name

%----------------------------------------------------------------------------------------

\begin{document}

\maketitle
\newpage

%----------------------------------------------------------------------------------------
%	TABLE OF CONTENTS
%----------------------------------------------------------------------------------------

%\setcounter{tocdepth}{1} % Uncomment this line if you don't want subsections listed in the ToC
%
%\newpage
%\tableofcontents
%\newpage

%----------------------------------------------------------------------------------------
%	PROBLEM 1
%----------------------------------------------------------------------------------------

% To have just one problem per page, simply put a \clearpage after each problem


\begin{figure*}

\centering
\subfloat[Addition]{

    \begin{tabular}{c|ccccc}
     			& $\bot$  		& $\mathbb{Z-}$ 	& $0$		 & $\mathbb{Z+}$ 	& $\top$ \\ \hline
    $\bot$  		& $\bot$			&  $\bot$   			&     $\bot$          	&	    $\bot$                    	&    $\bot$  \\
 $\mathbb{Z-}$ 	& $\bot$			&  $\mathbb{Z-}$    			&         $\mathbb{Z-}$      	&          $\top$    			&        $\top$         \\
    $0$    		& $\bot$			&   $\mathbb{Z-}$ 			&       $0$         	&            $\mathbb{Z+}$   			&     $\top$              \\
 $\mathbb{Z+}$&  $\bot$			&   $\top$   			&  $\mathbb{Z+}$              	&       $\mathbb{Z+}$       			&     $\top$            \\
    $\top$       	& $\bot$     				&      	$\top$		&        $\top$       	&            $\top$  			&     $\top$           \\ 
    
    \end{tabular}
    \label{tbl:add}
}~~~~~
\subfloat[Subtraction]{
    \begin{tabular}{c|ccccc}
     			& $\bot$  		& $\mathbb{Z-}$ 	& $0$		 & $\mathbb{Z+}$ 	& $\top$ \\ \hline
    $\bot$  		& $\bot$			&  $\bot$   			&     $\bot$          	&	    $\bot$                    	&    $\bot$  \\
 $\mathbb{Z-}$ 	& $\bot$			&  $\top$    			&         $\mathbb{Z-}$      	&          $\mathbb{Z-}$   			&        $\top$         \\
    $0$    		& $\bot$			&   $\mathbb{Z+}$ 			&       $0$         	&            $\mathbb{Z-}$   			&     $\top$              \\
 $\mathbb{Z+}$&  $\bot$			&   $\mathbb{Z+}$   			&  $\mathbb{Z+}$              	&       $\top$       			&     $\top$            \\
    $\top$       	& $\bot$     				&      	$\top$		&        $\top$       	&            $\top$  			&     $\top$           \\ \end{tabular}
\label{tbl:sub}
}

\subfloat[Multiplication]{
    \begin{tabular}{c|ccccc}
     			& $\bot$  		& $\mathbb{Z-}$ 	& $0$		 & $\mathbb{Z+}$ 	& $\top$ \\ \hline
    $\bot$  		& $\bot$			&  $\bot$   			&     $\bot$          	&	    $\bot$                    	&    $\bot$  \\
 $\mathbb{Z-}$ 	& $\bot$			&  $\mathbb{Z+}$    			&         $0$      	&          $\mathbb{Z-}$    			&        $\top$         \\
    $0$    		& $\bot$			&   $0$ 			&       $0$         	&            $0$   			&     $0$              \\
 $\mathbb{Z+}$&  $\bot$			&   $\mathbb{Z-}$   			&  $0$              	&       $\mathbb{Z+}$       			&     $\top$            \\
    $\top$       	& $\bot$     				&      	$\top$		&        $0$       	&            $\top$  			&     $\top$           \\ \end{tabular}
\label{tbl:mul}
}~~~~~
\subfloat[Division]{
    \begin{tabular}{c|ccccc}
     			& $\bot$  		& $\mathbb{Z-}$ 	& $0$		 & $\mathbb{Z+}$ 	& $\top$ \\ \hline
    $\bot$  		& $\bot$			&  $\bot$   			&     $\bot$          	&	    $\bot$                    	&    $\bot$  \\
 $\mathbb{Z-}$ 	& $\bot$			&  $\mathbb{Z+}$    			&         $\bot$      	&          $\mathbb{Z-}$    			&        $\top$         \\
    $0$    		& $\bot$			&   $0$ 			&       $\bot$         	&            $0$   			&     $0$              \\
 $\mathbb{Z+}$&  $\bot$			&   $\mathbb{Z-}$   			&  $\bot$              	&       $\mathbb{Z+}$       			&     $\top$            \\
    $\top$       	& $\bot$     				&      	$\top$		&        $\bot$       	&            $\top$  			&     $\top$           \\ \end{tabular}
\label{tbl:div}
}

\caption{Arithmetic Tables}
\label{fig:table}
\end{figure*}

%\begin{homeworkProblem}
\section{Arithmetic Operators}
\emph{In a separate PDF document generated from Latex, formalize the abstract arithmetic operators on the integer abstract domain (i.e., addition, subtraction, multiplication, and division) and prove that they are all monotone (hint: the easiest way to formalize operators on finite abstract domains is usually to give them as a table).
}
\\\\
For a function to be monotone, we must show that the function $f:\mathbb{S}\rightarrow \mathbb{S'}$, the following holds $\forall x,y \in S : x \sqsubseteq y \Rightarrow f(x) \sqsubseteq f(y)$
We define our abstraction function as $\alpha^{\#}:\mathbb{Z}\rightarrow \mathbb{Z^\#} = \mathbb{\{\bot,Z-,}0\mathbb{,Z+,\top\}}$ where $\top = \mathbb{Z}$.
We define $(x,y) \sqsubseteq (x',y')$ for $x,y,x',y' \in \mathbb{Z^\#} \iff x \sqsubseteq y$ and $x' \sqsubseteq y'$.  For the cases where $x = (\bot,*)$, the result $\bot$ is trivially $\sqsubseteq y, \forall y \in \mathbb{Z^\#}$, and similarly with $x = (*,\top)$, which yields $\top$.

For convenience, the table for each operation can be found in Figure \ref{fig:table}.

\subsection{Addition}
Thus, for $x,y \in \mathbb{Z}$ we can show a proof by cases.  
Since addition is commutative, without loss of generality, we denote $(x,y) \in (\mathbb{Z^\#},\mathbb{Z^\#})$ where $x \sqsubseteq y$.
Our addition function is $f^{+}:(\mathbb{Z^\#},\mathbb{Z^\#})\rightarrow\mathbb{Z^\#}$.
\paragraph{$x = (\mathbb{Z-},\mathbb{Z-})$}
\begin{itemize}
\item $y = (\mathbb{Z-},\mathbb{Z-}) \Rightarrow \alpha^{\#}(x) \sqsubseteq \alpha^{\#}(y) \Rightarrow \mathbb{Z-} \sqsubseteq \mathbb{Z-}$
\item $y = (\mathbb{Z-},\top) \Rightarrow \alpha^{\#}(x) \sqsubseteq \alpha^{\#}(0) \Rightarrow \mathbb{Z-} \sqsubseteq \top$
\item $y = (\top,\top) \Rightarrow \alpha^{\#}(x) \sqsubseteq \alpha^{\#}(y) \Rightarrow \mathbb{Z-} \sqsubseteq \top$
\end{itemize}
\paragraph{$x = (0,0)$}
\begin{itemize}
\item $y = (0,0) \Rightarrow \alpha^{\#}(x) \sqsubseteq \alpha^{\#}(0) \Rightarrow 0 \sqsubseteq 0$
\item $y = (0,\top) \Rightarrow \alpha^{\#}(x) \sqsubseteq \alpha^{\#}(y) \Rightarrow 0 \sqsubseteq \top$
\item $y = (\top,\top) \Rightarrow \alpha^{\#}(x) \sqsubseteq \alpha^{\#}(y) \Rightarrow 0 \sqsubseteq \top$
\end{itemize}
\paragraph{$x = (\mathbb{Z+},\mathbb{Z+})$}
\begin{itemize}
\item $y = (\mathbb{Z+},\mathbb{Z+}) \Rightarrow \alpha^{\#}(x) \sqsubseteq \alpha^{\#}(0) \Rightarrow \mathbb{Z+} \sqsubseteq \mathbb{Z+}$
\item $y = (\mathbb{Z+},\top) \Rightarrow \alpha^{\#}(x) \sqsubseteq \alpha^{\#}(y) \Rightarrow \mathbb{Z+} \sqsubseteq \top$
\item $y = (\top,\top) \Rightarrow \alpha^{\#}(x) \sqsubseteq \alpha^{\#}(y) \Rightarrow \mathbb{Z+} \sqsubseteq \top$
\end{itemize} $\square$

\subsection{Subtraction}
For subtraction, commutativity does not hold, so we will outline the possible cases below.
A similar logic holds for $\top$ and $\bot$ as it did it addition.
However note that $\forall x = (*,\top), (\top,*)  \alpha^{\#}(x) = \top$ and is thus trivially is will satisfy monotonicity for any $x' \sqsubseteq (*,\top)$ or $(\top,*)$.
Thus, the only remaining comparisons are where $(x,y) = (x',y')$, which also trivially hold.
A proof by cases as done in the addition would also be possible, but unnecessary. $\square$

\subsection{Multiplication}
A similar argument to our proof in subtraction and addition hold here, i.e., $\bot$ will always satisfy our requirement, and given that multiplication is commutative, we again can assume $(x,y) \Rightarrow x \sqsubseteq y$.
$(\mathbb{Z-},\top), (\mathbb{Z+},\top)$, and $(\top,\top)$ will also hold under the same logic, however 0 is a special case here.
However, the identity trivially holds, i.e.,$x = (0,\top), y = (0,\top) \Rightarrow 0 \sqsubseteq 0$ as does $x = (\top,\top), y = (\top,\top) \Rightarrow \top \sqsubseteq \top$.
Thus satisfying the requirement to be monotone. $\square$

\subsection{Division}
The same logic as before applies again to every case but $(\top,0)$ and $(0,\top)$.
However since the identity will trivially hold, and $\alpha^{\#}((\top,0)) = \bot \sqsubseteq \alpha^{\#}((\top,\top)) = \top$ and $\alpha^{\#}((0,\top)) = 0 \sqsubseteq \alpha^{\#}((\top,\top)) = \top$, we again have shown that the function is monotone.  $\square$







%\end{homeworkProblem}

\end{document}